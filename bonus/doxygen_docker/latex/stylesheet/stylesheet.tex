%% 
%% EPITECH PROJECT, 2024
%% my_zappy 
%% File description:
%% stylesheet.tex
%%

\documentclass{article}
\usepackage[utf8]{inputenc} % For Unicode characters
\usepackage{graphicx} % For including images
\usepackage{hyperref} % For hyperlinks
\usepackage{fancyhdr} % For custom headers and footers
\usepackage{lipsum} % For generating dummy text (remove in actual document)

% Custom header and footer
\pagestyle{fancy}
\fancyhf{} % Clear header and footer
\rhead{Zappy - Doxygen - 2024}
\lfoot{\thepage}
\rfoot{Zappy Game}

\begin{document}

% Title
\title{Zappy - Doxygen - 2024}
\author{Harleen Singh-Kaur, Thomas Lebouc, Eric Xu, Victor Yvon, Henry Letellier}
\date{}
\maketitle

% Main content
\section{Introduction}
The Zappy project is a network idle game in which AI's (provided with the project) will play on a defined width and height board.
The winning team is the first one where at least 6 players who reach the maximum elevation.
The following pages describe all the details and constraints.

During this game, the ai's are going to need to collect as many ressources possible.
When an ai has collected a certain amount of ressources, it can ascend (thus progress in the game) their vision distance increases.

There are 3 binaries: zappy_server (the server), zappy_ai (the ai), zappy_gui (for the window [a way of seeing how the game is unfolding])

% Dummy text for demonstration purposes
\lipsum[1-10]

% Footer
\clearpage
\thispagestyle{empty} % No header or footer on this page
\begin{center}
    \small \textcopyright{} 2024 Zappy Game. All rights reserved. \\
    \small This is a project created in the Epitech education environment.
    \small They were the ones who asked us to create the project.
\end{center}

\end{document}
